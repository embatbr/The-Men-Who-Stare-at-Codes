% Document number 1 of N, part of a concise course in Probability Theory and
% Statistics.
% References:
% [1] A First Course in Probability - 8th Edition - Sheldon Ross

\documentclass[a4paper,twocolumn]{article}

\usepackage{times}
\usepackage[utf8]{inputenc}
\usepackage[english]{babel}
\usepackage[a4paper,margin=2cm,columnsep=1cm]{geometry}
\usepackage{authblk}
\usepackage{titlesec}
\usepackage[pdftex]{graphicx}
\usepackage{mathtools}


\begin{document}

\graphicspath{{images/}}
\renewcommand{\abstractname}{\normalsize\bfseries\filcenter Abstract}
\titleformat*{\section}{\normalsize\bfseries}
\titleformat*{\subsection}{\small\bfseries}
\renewcommand{\refname}{\normalsize\bfseries\filcenter References}
\renewcommand{\figurename}{\small Figure}
\newcommand{\figureref}[1]{\textit{Figure \ref{fig:#1}}}
\newcommand{\equationref}[1]{\textit{Equation \ref{eq:#1}}}


\title{
    \textbf{Probability Theory and Statistics}\\
    \textit{\Large A Very Concise Approach}
}
\author{
    \textbf{Eduardo M. B. de A. Tenório}\\
    \small\texttt{embatbr@gmail.com}
}
\affil{
    \large{The Men Who Stare at Codes}\\
    \small\texttt{themenwhostareatcodes.wordpress.com}
}
\date{\today}

\maketitle


\section{Introduction}

According to the \textit{Encyclopaedia Britannica}, \textbf{probability theory} is ``a branch of mathematics concerned with the analysis of random phenomena" \cite{enc_brit_prob}. This science is used to measure the likeliness that an event will occur. \textbf{Statistics} is defined by \textit{The Oxford Dictionary of Statistical Terms} as ``the study of the collection, analysis, interpretation, presentation and organization of data" \cite{oxford_stat}.

A good example of probability and statistics as a tool is in election polls: researchers use the statistics of the voting population (schooling, region and etc.) to know the probability of victory for each candidate, by taking a sample, generalizing it and calculating the similarity to reality. Machine learning algorithms and insurance policies are also good examples of probability and statistics used to extract information.

This document is a working in progress, where I intend to present the basic topics of Probability Theory and Statistics. So, it will change with time. The main reference, will be the book \textbf{A First Course in Probability - 8th ed.} \cite{ross}, but probably other materials will be used. If the reader desires to go deeper, further readings are recommended.


\section{Combinatorial Analysis}

A communication system consisting of a transmitter, four antennas (signal repeaters) and a receiver is organized in a straight line. The system is called \textit{functional} as long as two consecutive antennas are not defective. If we have exactly $m$ of $n$ defective antennas, what is the probability that the resulting system will be functional? For the particular case where we have $n = 4$ and $m = 2$, there are 6 possible configurations, namely,

\begin{center}
0 1 1 0\\
0 1 0 1\\
1 0 1 0\\
0 0 1 1\\
1 0 0 1\\
1 1 0 0
\end{center}

\noindent where 1 means \textit{working} and 0, \textit{defective}. As the resulting system is functional in the first 3 arrangements and not functional in the remaining 3, it is clear that the probability of have a functional system is $\frac{3}{6} = \frac{1}{2}$. So, if we generalize it to undefined $m$ and $n$ values, our probability problem reduces to a counting problem. The mathematical theory of counting is formally known as \textbf{combinatorial analysis}.


\subsection{The Generalized Basic Principle of Counting}

If $r$ experiments that are to be performed are such that the first one may result in any of $n_1$ possible outcomes; and if, for each of these $n_1$ possible outcomes, there are $n_2$ possible outcomes of the second experiment; and if, for each of the possible outcomes of the first two experiments, there are $n_3$ possible outcomes of the third experiment; and if \dots , then there is a total of $n_1 n_2 \dotsm n_r = \prod_{i=1}^{r} n_i$ possible outcomes of the $r$ experiments \cite{ross}.\\

\noindent\textbf{\textit{Example.}} How many different 7-place license plates are possible if the first 3 places are to be occupied by letters and the final 4 by numbers?\\
\noindent\textbf{\textit{Solution.}} By the generalized basic principle of counting, the answer is $26 \cdot 26 \cdot 26 \cdot 10 \cdot 10 \cdot 10 \cdot 10 = 175,760,000$.


\subsection{Permutations}

In mathematics, \textbf{permutation} means ``rearranging the elements of a set in a sequence", and the number of possible sequences is a function of the set's size. Given a set of $n$ elements, we can \textit{permute} it's elements according to the formula:

\begin{center}
$P_n = n(n - 1)(n - 2) \dotsm 3 \cdot 2 \cdot 1 = n!$
\end{center}

\noindent As we are arranging the $n$ elements in $n$ positions, the first place can be chosen from $n$ possibilities, the second from $(n - 1)$ and so on, until the last with just one remaining option. We use the $!$ after a arbitrary number $n$ to denote the \textbf{factorial} of $n$.\\

\noindent\textbf{\textit{Example.}} In a group of 6 men and 4 women, how many possible arrangments can we have if \textbf{\textit{(a)}} put all in the same group, and \textbf{\textit{(b)}} divide by sex?\\
\noindent\textbf{\textit{Solution.}} \textbf{\textit{(a)}} We have 10 people to rearrange into 10 positions, so we just do $P_{10} = 10! = 3,628,800$.\\
\textbf{\textit{(b)}} Using the \textbf{basic principle} we have two different experiments (rearranging men and rearranging women). So, our result is just $P_6 P_4 = 6!4! = 720 \cdot 24 = 17,280$ possibilities.\\

If our permutation do not preserve beginning and end (called \textbf{circular permutation}), the formula is no longer $n!$ (simple permutation), but $(n - 1)!$. For example, if we permute $abc$ circularly, the results $abc$, $cab$ and $bca$ are equal. The same to $acb$, $bac$ and $cba$. For $n$ elements, as the first $n$ possibilities of the previous ``first position" are the same, we do

\begin{center}
$Pc_n = \frac{P_n}{n} = \frac{n!}{n} = (n - 1)!$
\end{center}

\noindent\textbf{Remember:} the order is still importante, but know the line does not have a beginning and an end, but a circular organization.


\subsection{$k$-permutations of $n$}

A permutation of $n$ elements in $k$ positions is also called an ``arrangement of $n$, $k$ to $k$" and can be symbolized by $A_{n,k}$. Given the simple permutation shown before ($P_n = n!$), if we permute the $n$ elements in $k$ positions, this means that the first position has $n$ possible values, the second has $(n -1)$ possibilities and so on, until the last position having $(n - k)$ possibilities.\\

\noindent\textbf{\textit{Example.}} We have 7 athletes in a competition and want to rank them in the three best positions: gold, silver and bronze medals.\\
\noindent\textbf{\textit{Solution.}} The gold medal can be won by all 7 athletes. Ranked the first place, the second has 6 options and the third, 5. This means that $A_{7,3} = 7 \cdot 6 \cdot 5 = 210$.\\

\noindent So, we can generalize it, think a little and have the simple formula $A_{n,k} = \frac{n!}{(n - k)!}$ for an arrangement.


\subsection{Combinations}

If we join the ideas of a circular permutation (no order) and $k$-permutation of $n$ we develop a new interesting idea. In the previous example (the competition), chosen the first three places, we can permute them in $3! = 6$ ways. If we abolish the different medals and just give a ``well done" medal to all three, the six different configurations involving these particular athletes are the same and we can count them as just one. The same to the other 209 configurations. So, now we have $\frac{A_{7,3}}{6} = 35$. This is named \textbf{combination},

\begin{center}
$C_{n,k} = \binom{n}{k} = \frac{A_{n,k}}{k!} = \frac{n!}{k!(n - k)!}$
\end{center}

\noindent and is read ``combination of $n$, $k$ to $k$". Remember that $r \leq n$ always (but I shouldn't need to say that).

Now you have the tools to answer the question from the beginning of this section (I will skip this one because I am lazy and already know it) and go deeper into this subject, if you desire. Try to read something about \textbf{binomial theorem} and to generalize the combinatorial analysis to \textbf{multinomial coefficients}.


\begin{thebibliography}{9}
    \bibitem{enc_brit_prob}
         ``Probability theory, Encyclopaedia Britannica".
         Britannica.com.
         Retrieved 2014-10-11

    \bibitem{oxford_stat}
        Dodge, Y. (2006)
        \textit{The Oxford Dictionary of Statistical Terms}, OUP.
        ISBN 0-19-920613-9

    \bibitem{ross}
        Ross, Sheldon M.,
        2010,
        ``A First Course in Probability", 8th ed.,
        Prentice Hall
\end{thebibliography}


\end{document}